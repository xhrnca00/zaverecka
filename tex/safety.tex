\documentclass[main.tex]{subfiles}

\begin{document}
\kapitola{Bezpečnost programů}

Hlavní důvod, proč Rust nahrazuje C++, je jeho lepší bezpečnost. C++ jako takové nepatří
mezi nejbezpečnější jazyky~-- je poměrně jednoduché v~něm vytvořit memory leak a celkově
je zde hodně věcí, na které si musí programátor dávat pozor.

\sekce{Paměť}

Asi nejunikátnější z~věcí, které Rust přináší, je tzv. \emph{vlastnictví} (angl.
ownership). Historicky se k~udržování bezpečnosti paměti používaly dva různé způsoby:
garbage collection (již nepotřebná paměť se \uv{sama uklidí}) a manuální
alokace/dealokace. Garbage collection je pomalejší, ale zato zaručeně bezpečný. Díky
konceptu vlastnictví \emph{Rust garantuje paměťovou bezpečnost} bez zpomalení garbage
collectorem. \cite[kapitola\,4]{thebook}

\podsekce{Pravidla vlastnictví}

Tento koncept se řídí poměrně jednoduchými pravidly \cite[sekce\,4.1]{thebook}:
\begin{itemize}
    \item Každá hodnota má \emph{právě jednoho} vlastníka
    \item Když skončí životnost vlastníka, končí i~životnost hodnoty (a tedy se hodnota
          bezpečně vyčistí/dealokuje apod.)
\end{itemize}

Toto také souvisí s~principem RAII (Resource Acquisition Is Initialization), znamým
i~z~C++ \cite[language/raii]{cppreference}. Když tedy vznikne proměnná, vzniká zároveň
s~ní i~hodnota (např. se alokuje na haldě).

Jelikož se jedná o~zcela unikátní koncept, je nutné si zvyknout na jiný mentální model.
Pojďme se znovu podívat na příklad move sémantiky (Obr.\,\ref{move semantics showcase})
z~dřívější kapitoly, tentokrát z~pohledu vlastnictví:

\obrazek
\begin{rustcode*}{highlightlines={3-3},highlightcolor=m-hi-error}
    let a = vec![1, 2, 3];
    let b: Vec<i32> = a; // move `a` into `b`
    for elem in a {
        println!("{}", elem);
    }
\end{rustcode*}
\endobrl{Připomenutí ukázky move sémantiky}{ownership showcase}

Ze začátku vlastní vektor (neboli hodnotu) proměnná \irust{a}. Na dalším řádku ale
\emph{předáváme vlastnictví vektoru proměnné \irust{b}}. Když tedy později chceme použít
proměnnou \irust{a}, nejde to, protože \emph{\irust{a} již vektor nevlastní}.

\podsekce{Půjčování}

Vlastnictví představuje nový mentální model a vysvětluje move sémantiku, ale jak do toho
zapadají reference? Zde přichází koncept \emph{půjčování} (angl. borrowing). Jeho pravidla
kontroluje část kompilátoru jménem \emph{borrow checker}.

Když vytváříme referenci, \uv{půjčujeme si hodnotu}. \uv{Dívat} na hodnotu se může
libovolný počet proměnných, problém by ale nastal, kdyby někdo hodnotu měnil. Provádět
změny proto může pouze jedna proměnná a to pouze v~případě, že nikdo jiný nemá hodnotu
půjčenou na prohlížení. Formálněji \emph{V~jakýkoliv čas jde mít buď jednu
    modifikovatelnou referenci (\irust{&mut T}), nebo libovolný počet nemodifikovatelných
    referencí (\irust{&T}).} \cite[sekce\,4.2]{thebook}

\obrazek
\begin{rustcode*}{highlightlines={3-3},highlightcolor=m-hi-error}
    let mut value = 5;
    let reference = &value;
    let mut mutable_reference = &mut value;
    println!("{}", reference);
\end{rustcode*}
\newline
\vlozobr{images/borrow-mut-compile-error.png}{0.26}
\endobrl{Ukázka půjčování}{borrowing showcase}

V~ukázce vytváříme nejdřív konstantní referenci a potom modifikovatelnou referenci.
Tiskneme ale pomocí první z~nich~-- obě tedy existují, což je problém. Kdybychom prohodili
pořadí vytváření referencí nebo tiskli pomocí modifikovatelné reference (nebo přímo
proměnné s~hodnotou), nejednalo by se o~error.

Druhým pravidlem půjčování je, že \emph{všechny refrence musí být validní}. Občas
(například když funkce vrací referenci) jsou k~zajištění tohoto pravidla nutné tzv.
životnostní anotace~-- do těch ale nebudeme zabíhat.

Příklad 2.\,pravidla můžeme vidět na obr.\,\ref{invalid references showcase}. Zde měníme,
na co reference ukazuje (používáme tedy reference více jako ukazatele v~C++). Problémem
ale je, že proměnná, na kterou se odkazujeme (kterou jsme si \uv{půjčili}) již neexistuje,
tedy je naše reference již nevalidní. Kdybychom referenci dále nepoužili, nejednalo by se
o~error.

\obrazek
\begin{rustcode*}{highlightlines={5-5},highlightcolor=m-hi-error}
    let value = 5;
    let mut reference: &u64 = &value;
    {
        let another = 30;
        reference = &another;
    }
    println!("{}", reference);
\end{rustcode*}
\newline
\vlozobr{images/borrow-invalid-compile-error.png}{0.26}
\endobrl{Ukázka nevalidních refrencí}{invalid references showcase}

\podsekce{Chytré ukazatele a halda}

Zatím jsme (kromě vektorů) alokovali pouze na zásobníku. Jak ale alokujeme na haldě?
Použijeme tzv. chytré ukazatele. Mohli bychom alokovat i~manuálně, ale to by vyžadovalo
\irust{unsafe} bloky (o~těch později) a bylo potenciálně nebezpečné. Proto v~Rustu
\emph{nikdy} nealokujeme manuálně, vždy používáme chytré ukazatele.

Chytré ukazatele v~rámci svého destruktoru dealokují i~svá data (k~modifikaci destruktoru
manuálně implementujeme \irust{Drop} trait). Nejčastěji používané chytré ukazatele jsou
\irust{Box<T>}, \irust{Rc<T>}\footnote{~Zkratka znamená \uv{reference counted}} a
\irust{Weak<T>}. Jejich ekvivalentem jsou \icpp{std::unique_ptr<T>},
\icpp{std::shared_ptr<T>} a \icpp{std::weak_ptr<T>}. Existují ale i~další chytré
ukazatele. \cite[kapitola\,15]{thebook}

Důležitý ukazatel je \irust{RefCell<T>}. Umožňuje nám totiž implementovat kód, který by
jinak nebyl možný bez \irust{unsafe} bloků. Umí dynamicky, během běhu programu, plnit
úkol borrow checkeru. Používáme jej, když by kompilátoru nebylo jasné, že jsme splnili
pravidla půjčování.

I~s~pomocí chytrých ukazatelů může vzniknout referenční cyklus a vytvořit memory leak!

Ukázka na obr.\,\ref{smart pointers showcase} by měla být jednoznačná. Pomocí \irust{use}
nejdříve importujeme \irust{Rc<T>} a \irust{Weak<T>}. \irust{Box<T>} importovat nemusíme,
protože je dostupný vždy. Vytváříme číslici~5 alokovanou na haldě. Do další proměnné
ukládáme \irust{Rc} se stejnou hodnotou jako ta, na kterou ukazuje \irust{Box}. Dále
z~\irust{Rc} vytváříme jednu silnou a jednu slabou referenci. Na posledních řádcích
vypisujeme počet referencí sdíleného ukazatele. Typy jsou přidány pouze pro pohodlí
čtenáře, na funkčnosti programu nic nemění.

\obrazek
\begin{rustcode}
    use std::rc::{Rc, Weak};

    fn main() {
        let boxed_value: Box<i32> = Box::new(5);
        let rc_value: Rc<i32> = Rc::new(*boxed_value);
        let strong_reference: Rc<i32> = Rc::clone(&rc_value);
        let weak_reference: Weak<i32> = Rc::downgrade(&rc_value);
        println!("{}", Rc::strong_count(&rc_value)); // 2
        println!("{}", Rc::weak_count(&rc_value)); // 1
    }
\end{rustcode}
\endobrl{UKázka chytrých ukazatelů}{smart pointers showcase}

\podsekce{Unsafe bloky}

S~pamětí souvisí i~klíčové slovo \irust{unsafe}. V~blocích označených tímto slovem můžeme,
kromě klasického \uv{safe} Rustu, provádět následujících 5\,věcí:
\begin{enumerate}
    \item \emph{dereferencovat ukazatele}
    \item volat unsafe metody
    \item \emph{číst nebo modifikovat statickou měnitelnou proměnnou} (\irust{static mut})
    \item implementovat unsafe trait
    \item přistoupit k~hodnotám unionu (abychom mohli interagovat s~uniony jazyka~C)
\end{enumerate}

Unsafe kód se často nepoužívá, ale jedná se o~způsob jak zvolnit pravidla Rustu, například
pro vytvoření lepší API nebo zlepšení výkonu. Některé objekty ve standardní knihovně
(třeba chytré ukazatele) unsafe bloky používají. \cite[sekce\,19.1]{thebook}

\sekce{Práce s~errory}
\begin{itemize}
    \item proč exceptions bad
    \item panic
    \item Ok/Err + Some/None
    \item ? operator
\end{itemize}
\sekce{Paralelní programování}
\begin{itemize}
    \item read/write safety
    \item traits are cool
    \item deadlock still possible
\end{itemize}

\end{document}
