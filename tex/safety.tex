\documentclass[main.tex]{subfiles}

\begin{document}
\kapitola{Bezpečnost programů}

Hlavní důvod, proč Rust nahrazuje C++, je jeho lepší bezpečnost. C++ jako takové nepatří
mezi nejbezpečnější jazyky~-- je poměrně jednoduché v~něm vytvořit memory leak a celkově
je zde hodně věcí, na které si musí programátor dávat pozor.

\sekce{Paměť}

Asi nejunikátnější z~věcí, které Rust přináší, je tzv. \emph{vlastnictví} (angl.
ownership). Historicky se k~udržování bezpečnosti paměti používaly dva různé způsoby:
garbage collection (již nepotřebná paměť se \uv{sama uklidí}) a manuální
alokace/dealokace. Garbage collection je pomalejší, ale zato zaručeně bezpečný. Díky
konceptu vlastnictví \emph{Rust garantuje paměťovou bezpečnost} bez zpomalení garbage
collectorem. \cite[kapitola\,4]{thebook}

\podsekce{Pravidla vlastnictví}

Tento koncept se řídí poměrně jednoduchými pravidly \cite[sekce\,4.1]{thebook}:
\begin{itemize}
    \item Každá hodnota má \emph{právě jednoho} vlastníka
    \item Když skončí životnost vlastníka, končí i~životnost hodnoty (a tedy se hodnota
          bezpečně vyčistí/dealokuje apod.)
\end{itemize}

Toto také souvisí s~principem RAII (Resource Acquisition Is Initialization), znamým
i~z~C++ \cite[language/raii]{cppreference}. Když tedy vznikne proměnná, vzniká zároveň
s~ní i~hodnota (např. se alokuje na haldě).

Jelikož se jedná o~zcela unikátní koncept, je nutné si zvyknout na jiný mentální model.
Pojďme se znovu podívat na příklad move sémantiky (Obr.\,\ref{move semantics showcase})
z~dřívější kapitoly, tentokrát z~pohledu vlastnictví:

\obrazek
\begin{rustcode*}{highlightlines={3-3},highlightcolor=m-hi-error}
    let a = vec![1, 2, 3];
    let b: Vec<i32> = a; // move `a` into `b`
    for elem in a {
        println!("{}", elem);
    }
\end{rustcode*}
\endobrl{Připomenutí ukázky move sémantiky}{ownership showcase}

Ze začátku vlastní vektor (neboli hodnotu) proměnná \irust{a}. Na dalším řádku ale
\emph{předáváme vlastnictví vektoru proměnné \irust{b}}. Když tedy později chceme použít
proměnnou \irust{a}, nejde to, protože \emph{\irust{a} již vektor nevlastní}.

\podsekce{Půjčování}

Vlastnictví představuje nový mentální model a vysvětluje move sémantiku, ale jak do toho
zapadají reference? Zde přichází koncept \emph{půjčování} (angl. borrowing). Jeho pravidla
kontroluje část kompilátoru jménem \emph{borrow checker}.

Když vytváříme referenci, \uv{půjčujeme si hodnotu}. \uv{Dívat} na hodnotu se může
libovolný počet proměnných, problém by ale nastal, kdyby někdo hodnotu měnil. Provádět
změny proto může pouze jedna proměnná a to pouze v~případě, že nikdo jiný nemá hodnotu
půjčenou na prohlížení. Formálněji \emph{V~jakýkoliv čas jde mít buď jednu
    modifikovatelnou referenci (\irust{&mut T}), nebo libovolný počet nemodifikovatelných
    referencí (\irust{&T}).} \cite[sekce\,4.2]{thebook}

\obrazek
\begin{rustcode*}{highlightlines={3-3},highlightcolor=m-hi-error}
    let mut value = 5;
    let reference = &value;
    let mut mutable_reference = &mut value;
    println!("{}", reference);
\end{rustcode*}
\newline
\vlozobr{images/borrow-mut-compile-error.png}{0.26}
\endobrl{Ukázka půjčování}{borrowing showcase}

V~ukázce vytváříme nejdřív konstantní referenci a potom modifikovatelnou referenci.
Tiskneme ale pomocí první z~nich~-- obě tedy existují, což je problém. Kdybychom prohodili
pořadí vytváření referencí nebo tiskli pomocí modifikovatelné reference (nebo přímo
proměnné s~hodnotou), nejednalo by se o~chybu.

Druhým pravidlem půjčování je, že \emph{všechny refrence musí být validní}. Občas
(například když funkce vrací referenci) jsou k~zajištění tohoto pravidla nutné tzv.
životnostní anotace~-- do těch ale nebudeme zabíhat.

Příklad 2.\,pravidla můžeme vidět na obr.\,\ref{invalid references showcase}. Zde měníme,
na co reference ukazuje (používáme tedy reference více jako ukazatele v~C++). Problémem
ale je, že proměnná, na kterou se odkazujeme (kterou jsme si \uv{půjčili}) již neexistuje,
tedy je naše reference již nevalidní. Kdybychom referenci dále nepoužili, nejednalo by se
o~chybu.

\obrazek
\begin{rustcode*}{highlightlines={5-5},highlightcolor=m-hi-error}
    let value = 5;
    let mut reference: &u64 = &value;
    {
        let another = 30;
        reference = &another;
    }
    println!("{}", reference);
\end{rustcode*}
\newline
\vlozobr{images/borrow-invalid-compile-error.png}{0.26}
\endobrl{Ukázka nevalidních refrencí}{invalid references showcase}

\podsekce{Chytré ukazatele a halda}

Zatím jsme (kromě vektorů) alokovali pouze na zásobníku. Jak ale alokujeme na haldě?
Použijeme tzv. chytré ukazatele. Mohli bychom alokovat i~manuálně, ale to by vyžadovalo
\irust{unsafe} bloky (o~těch později) a bylo potenciálně nebezpečné. Proto v~Rustu
\emph{nikdy} nealokujeme manuálně, vždy používáme chytré ukazatele.

Chytré ukazatele v~rámci svého destruktoru dealokují i~svá data (k~modifikaci destruktoru
manuálně implementujeme \irust{Drop} trait). Nejčastěji používané chytré ukazatele jsou
\irust{Box<T>}, \irust{Rc<T>}\footnote{~Zkratka znamená \uv{reference counted}} a
\irust{Weak<T>}. Jejich ekvivalentem jsou \icpp{std::unique_ptr<T>},
\icpp{std::shared_ptr<T>} a \icpp{std::weak_ptr<T>}. Existují ale i~další chytré
ukazatele. \cite[kapitola\,15]{thebook}

Důležitý ukazatel je \irust{RefCell<T>}. Umožňuje nám totiž implementovat kód, který by
jinak nebyl možný bez \irust{unsafe} bloků. Umí dynamicky, během běhu programu, plnit
úkol borrow checkeru. Používáme jej, když by kompilátoru nebylo jasné, že jsme splnili
pravidla půjčování.

I~s~pomocí chytrých ukazatelů může vzniknout referenční cyklus a vytvořit memory leak!

Ukázka na obr.\,\ref{smart pointers showcase} by měla být jednoznačná. Pomocí \irust{use}
nejdříve importujeme \irust{Rc<T>} a \irust{Weak<T>}. \irust{Box<T>} importovat nemusíme,
protože je dostupný vždy. Vytváříme číslici~5 alokovanou na haldě. Do další proměnné
ukládáme \irust{Rc} se stejnou hodnotou jako ta, na kterou ukazuje \irust{Box}. Dále
z~\irust{Rc} vytváříme jednu silnou a jednu slabou referenci. Na posledních řádcích
vypisujeme počet referencí sdíleného ukazatele. Typy jsou přidány pouze pro pohodlí
čtenáře, na funkčnosti programu nic nemění.

\obrazek
\begin{rustcode}
    use std::rc::{Rc, Weak};

    fn main() {
        let boxed_value: Box<i32> = Box::new(5);
        let rc_value: Rc<i32> = Rc::new(*boxed_value);
        let strong_reference: Rc<i32> = Rc::clone(&rc_value);
        let weak_reference: Weak<i32> = Rc::downgrade(&rc_value);
        println!("{}", Rc::strong_count(&rc_value)); // 2
        println!("{}", Rc::weak_count(&rc_value)); // 1
    }
\end{rustcode}
\endobrl{UKázka chytrých ukazatelů}{smart pointers showcase}

\podsekce{Unsafe bloky}

S~pamětí souvisí i~klíčové slovo \irust{unsafe}. V~blocích označených tímto slovem můžeme,
kromě klasického \uv{safe} Rustu, provádět následujících 5\,věcí:
\begin{enumerate}
    \item \emph{dereferencovat ukazatele}
    \item volat unsafe metody
    \item \emph{číst nebo modifikovat statickou měnitelnou proměnnou} (\irust{static mut})
    \item implementovat unsafe trait
    \item přistoupit k~hodnotám unionu (abychom mohli interagovat s~uniony jazyka~C)
\end{enumerate}

Unsafe kód se často nepoužívá, ale jedná se o~způsob jak zvolnit pravidla Rustu, například
pro vytvoření lepší API nebo zlepšení výkonu. Některé objekty ve standardní knihovně
(třeba chytré ukazatele) unsafe bloky používají. \cite[sekce\,19.1]{thebook}

\sekce{Zpracování chyb}

V~C++ jsme zvyklí vrhat chybové objekty pomocí \icpp{throw} a chytat je pomocí
\icpp{catch}. Tento přístup je špatně sledovatelný, jelikož bez pohledu na dokumentaci
nevíme, která funkce hází jakou chybu. Také můžeme mít problémy, když chybu zachytí nějaká
centrálnější funkce a nevíme, v~jakém stavu je náš program. Rust proto má zcela jiný
přístup k~práci s~chybami.

\podsekce{Nenávratné chyby, aneb panika!}

Rust má více typů chyb \cite[kapitola\,9]{thebook}. První z~nich jsou tzv. nenávratné
chyby. Těmto chybám se říká panika (angl. panic). Vyvolat je můžeme explicitně pomocí
makra \irust{panic!} (do argumentu makra dáme formátový řetězec, který se zobrazí jako
chybová hláška), nebo nějakou akcí, jako je například dělení nulou nebo špatná indexace
do řetězce (a tedy vznik neplatné UTF-8 sekvence). \cite[sekce\,9.1]{thebook} Panika se
používá, když se nemáme jak vyrovnat s~chybou.

\obrazek
\begin{rustcode}
    fn main() {
        panic!("crash and burn");
    }
\end{rustcode}
\newline
\vlozobr{images/panic-macro-panic.png}{0.26}
\endobrl{Ukázka paniky \obrzdroj{\cite[sekce\,9.1]{thebook}}}{panic showcase}

\podsekce{Obnovitelné chyby}

Druhým typem chyb jsou obnovitelné chyby. Funkce, ve kterých se může nějaká taková chyba
stát nevracejí přímo návratový typ~\texttt{T}, ale vracejí buď \irust{Option<T>}, nebo
\irust{Result<T, E>}, kde \texttt{E} je typ chyby. Tyto dva typy jsou enumy.
\cite[sekce\,9.2]{thebook}

Enum \irust{Option<T>} má dvě varianty: \irust{Some(T)} a \irust{None}. Používá se, když
nevíme, zda funkce vrátí hodnotu, nebo ne. \emph{Tento typ je také náhrada za hodnotu
    \texttt{null}, která v~Rustu neexistuje}.

\irust{Result<T, E>} má rovněž dvě varianty: \irust{Ok(T)} a \irust{Err(E)}. Tento typ
použijeme v~případě, že někde může nastat chyba, kterou chceme sdělit volajícímu. Typ
\texttt{E} může být cokoliv~-- enum (například u~I/O operací), řetězec,\dots

\podsekce{Práce s~chybovými hodnotami}

\emph{Pokud chceme získat typ~T, musíme jej dostat z~vrácené hodnoty}. Díky tomu jsme,
jakožto volající, \emph{nuceni} něco s~potenciální chybou udělat. Samozřejmě můžeme vždy použít \irust{match}, ale díky velkému množství pomocných metod to většinou není potřeba.
Jelikož jsou metody na obou typech funkčně ekvivalentní, budeme se dále zabírat pouze
typem \irust{Result}.

Asi nejlehčí, co můžeme udělat, je použít metodu \irust{unwrap}. Tato metoda nám vrátí typ
\texttt{T}. Pokud se ale jedná o~chybovou variantu, spustí se panika a program spadne.
Další metody, jako například \irust{map} a \irust{and_then} nám dovolují řetězit funkce
vracející \irust{Result}. Pokud chceme hodnotu \texttt{T} nebo nějakou výchozí hodnotu,
máme metodu \irust{unwrap_or}. \cite[result]{ruststd}

\obrazek
\begin{rustcode}
    // match
    let result: Result<i32, String> = Ok(200);
    match &result {
        Ok(code) => println!("Code: {}", code),
        Err(error) => println!("Error: {}", error),
    }
    // unwrap
    let code = result.unwrap();
    println!("Code: {}", code);
    // map
    let mut mode: Result<i32, String> = Ok(4);
    mode = mode.map(|c| c + 1);
    // unwrap\_or
    let mode = mode.unwrap_or(-1);
\end{rustcode}
\endobrl{Ukázka zpracování chyb}{result showcase}

\podsekce{Operátor ?}

Protože často chceme ve funkci řešit pouze případ bez chyby a předat chybu volajícímu,
vznikl operátor \irust{?}. Pokud naše funkce vrací stejný typ, jako námi volaná
funkce\footnote{~? ve skutečnosti umí provádět i~lehké typové konverze}, můžeme za
závorky přidat \irust{?}. Tento otazník vrátí chybovou hodnotu, pokud ji dostane; jinak
pokračuje dál. Výhodou je, že při programování funkce se staráme pouze o~bezchybovou
cestu. \cite[sekce\,9.2]{thebook}

Otazník můžeme použít i~v~\irust{main} funkci, pokud změníme její návratový typ
z~implicitního \irust{()} (prázdný typ) na \irust{Result<(), E>} a na jejím konci vrátíme
\irust{Ok(())}. Pokud v~tomto případě \irust{main} vrátí \irust{Err} variantu, vypíše se.

Na obr.\,\ref{? operator showcase} můžeme vidět příklad používající \irust{?}.

\obrazek
\begin{rustcode}
fn error_fn() -> Result<i32, String> {
    Ok(5)
}

fn do_stuff(num: i32) -> Result<i32, String> {
    let result = error_fn()?;
    if num == result {
        return Err(format!("{} is equal to {}", num, result));
    }
    Ok(result + num)
}

fn main() -> Result<(), String> {
    let code = do_stuff(5)?;
    println!("Code: {}", code);
    Ok(())
}
\end{rustcode}
\newline
\vlozobr{images/return-error-panic.png}{0.27}
\endobrl{Ukázka použití \irust{?}}{? operator showcase}

\sekce{Paralelní programování}

Paralelní programování v~jazyce Rust se velmi liší od C++. Kód se totiž nezkompiluje,
dokud není bezpečný. Rust této vlastnosti říká \emph{fearless concurrency}
\cite[kapitola\,16]{thebook}.

\podsekce{Syntax vytváření vláken}

Vlákna se, podobně jako v~C++, vytvářejí pomocí \irust{thread::spawn}. Objekt, který
funkce vrací má metodu \irust{join}, která vrací \irust{Result} (buď s~návratovou
hodnotou vlákna, nebo s~chybou, se kterou vlákno spadlo).

Funkce \irust{thread::spawn} bere  parametr, který implementuje trait \irust{FnOnce}
(tedy je funkcí, kterou jde zavolat aspoň jednou). Buď zde můžeme napsat přímo název
funkce, nebo můžeme vytvořit tzv. closure (anonymní funkci).

Syntax vytváření anonymní funkce je poměrně složitý. Mezi svislé čáry (\texttt{|})
napíšeme parametry. Za čárami následuje tělo, většinou ohraničené složenými závorkami
(pokud se nejedná o~jediný výraz). Při vytváření funkcí pro jiná vlákna před první
svislou čáru píšeme klíčové slovo \irust{move}, které říká, že všechny vnější proměnné,
které anonymní funkce používá, se mají přesunout do funkce. Toto je nutné, protože jinak
kompilátor nemůže poznat, zda jsou reference validní po celou dobu běhu vlákna.
\cite[sekce\,13.1]{thebook}

Na obr.\,\ref{simple thread creation showcase} můžeme vidět ukázku vytvoření vlákna, které
píše do standardního výstupu.

\obrazek
\begin{rustcode}
    use std::thread;

    fn main() {
        let value = 5;
        let handle = thread::spawn(move || {
            println!("Hello from a thread!");
            println!("Value: {}", value);
        });
        handle.join().unwrap();
        println!("Finished!");
    }
\end{rustcode}
\endobrl{Ukázka vytvoření vlákna}{simple thread creation showcase}

\podsekce{Atomické proměnné}

Rust má atomické typy jako \irust{AtomicU32}, podobně jako C++ (\icpp{atomic_uint32_t}).
Operace s~těmito hodnotami berou parametr typu \irust{Ordering}, jejichž hodnoty jsou
stejné jako řazení z~C++20 (\icpp{std::memory_order}) \cite[sync/atomic]{ruststd}.

\podsekce{Značící traity}

Abychom mohli mezi vlákny posílat hodnoty, musejí implementovat \irust{Send}. Toto
zaručuje, že se více vláken nedostane k~hodnotě, která na to není připravena. Například se
jedná o~ukazatel \irust{Rc<T>}, který nepoužívá atomické proměnné pro udržování
referenčních počtů. Pokud všechny části typu~\irust{T} implementují \irust{Send},
implementuje jej i~typ~\irust{T}. \cite[marker/trait.Send]{ruststd}

Pro přístup z~více vláken je potřeba trait \irust{Sync}. Typ~\irust{T} implementuje
\irust{Sync}, pokud \irust{&T} implementuje \irust{Send}.
\cite[marker/trait.Sync]{ruststd}

Tyto dva traity umožňují kompilátoru zajistit bezpečnost jen pomocí generik.

\podsekce{Chytré ukazatele}

Pokud chceme z~více vláken modifikovat jednu proměnnou, musíme použít vzájemné vyloučení
(angl. mutual exclusion, zkráceně mutex). Na rozdíl od C++ se ale nejedná o~\uv{hodnotu},
ale o~chytrý ukazatel. Metoda \irust{lock} pak vrací hodnotu\footnote{~ve skutečnosti
    vrací \irust{Result}. O~možnost \irust{Err} se jedná tehdy, když nějaké vlákno spadlo,
    zatímco mělo vlastnictví zámku.
}. Abychom ale mohli použít hodnotu \irust{Mutex<T>}, musíme ji přesunout do vlákna. Když
ji ale přesuneme, ztratíme možnost ji poslat do dalšího~-- je tedy potřeba jej zabalit
do sdíleného ukazatele.

Tento ukazatel je \irust{Arc<T>}\footnote{~atomically reference counted}. Má stejnou
funkcionalitu jako \irust{Rc<T>}, pouze používá atomické proměnné k~udržování počtů
referencí. \cite[sekce\,16.3]{thebook}

Na následujícím obrázku je krátký program, který vytvoří 3\,vlákna, z~nichž každé přičte
jedničku k~hodnotě uložené v~proměnné \irust{data}. Kdybychom nepoužili \irust{Arc<T>} a
\irust{Mutex<T>}, \emph{kód by se nezkompiloval}.

\obrazek
\begin{rustcode}
    use std::sync::{Arc, Mutex};
    use std::thread;

    fn main() {
        let mut data = Arc::new(Mutex::new(0));
        let mut handles = vec![];
        for x in 0..3 {
            let data_ref = Arc::clone(&data);
            let handle = thread::spawn(move || {
                let mut data = data_ref.lock().unwrap();
                *data += 1;
            });
            handles.push(handle);
        }
        for handle in handles {
            handle.join().unwrap();
        }
        let data: u64 = *data.lock().unwrap();
        println!("{}", data); // 3
    }
\end{rustcode}
\endobrl{Ukázka používání měnění proměnné z~více vláken}{threading showcase}

\podsekce{Message passing}

Existuje i~druhý typ paralelního programování, kde se používá posílání \uv{zpráv} (hodnot)
mezi vlákny. Protože standardní knihovna C++ tento způsob nepodporuje (nemá kanály se
synchronizací \cite{cppreference}), nebudeme do tohoto způsobu moc zabývat. V~Rustu jde
použít primitiv \irust{mpsc} (multiple producer, single consumer).
\cite[sekce\,16.2]{thebook}

\podsekce{Uváznutí}

Rust sice garantuje synchronizaci a správné posílání mezi vlákny, ale \emph{negarantuje,
    že vlákna nebudou čekat na sebe navzájem
} (uváznutí, nebo také deadlock). \cite[sekce\,16.3]{thebook}

\end{document}
