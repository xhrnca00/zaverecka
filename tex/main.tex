\documentclass[
    % -FIXME: add twoside back when printing
    twoside,
    12pt
]{article}
% pro tisk po jedné straně papíru je potřebné odstranit volbu twoside
% !TEX program = xelatex

%* packages
% \usepackage{indentfirst} % indent first paragraph
\usepackage{xdipp}
\usepackage[
    cachedir=./minted/,
    outputdir=./TempTeX/,
]{minted}
\usepackage{xcolor}
\usepackage{hyperref}
% \newcommand{\url}[1]{\texttt{#1}}

%* minted options
\setminted{autogobble,
    % linenos, % line numbers
    % xleftmargin=25pt, % space for line numbers to fit inside
    mathescape,
    texcomments,
    obeytabs,
    style=tango, % algol_nu, % tango,
    bgcolor=m-bg,
    highlightcolor=m-hi-default,
    tabsize=4,
}
\newminted[rustcode]{rust}{}
\newmint[rust]{rust}{}
\newmintedfile[rustfile]{rust}{stripall}
\newmintinline[irust]{rust}{stripall,bgcolor={}}
\newminted[cppcode]{cpp}{}
\newmint[cpp]{cpp}{}
\newmintedfile[cppfile]{cpp}{stripall}
\newmintinline[icpp]{cpp}{stripall,bgcolor={}}
\newminted[asmcode]{nasm}{}
\newmint[asm]{nasm}{}
\newmintedfile[asmfile]{nasm}{stripall}
\newmintinline[iasm]{nasm}{stripall,bgcolor={}}

%* patch minted
\makeatletter
% replace \medskip before and after the box with nothing, i.e., remove it
\patchcmd{\minted@colorbg}{\medskip}{}{}{}
\patchcmd{\endminted@colorbg}{\medskip}{}{}{}
\makeatother

%* setting options
% \cestina % implicitní
% \slovencina
% \english
% \pismo{LModern}
\pismo{Lora} % nic (=LModern), Academica, Baskerville, Bookman, Cambria, Comenia, Constantia, Palatino, Times
% \dvafonty % implicitně je \jedenfont
% \technika
% \beletrie % implicitní
\popiskyzkr{}
% \popisky % implicitní
\pagestyle{headings} % implicitní
\cislovat{2}
% \bakalarska % implicitní
\zaverecna{}
% \diplomova
% \disertacni
\brokenpenalty=10000

%* debugging
% \usepackage{showframe}
\overfullrule=4cm % shows slight overfulls as big black rectangles

%* hyphenation
\hyphenation{troj-úhel-ník}

%* colors
% courtesy of Tailwind CSS (*-100)
\definecolor{m-bg}{HTML}{f3f4f6}
\definecolor{m-hi-default}{HTML}{ecfccb}
\definecolor{m-hi-info}{HTML}{dbeafe}
\definecolor{m-hi-alert}{HTML}{fef3c7}
\definecolor{m-hi-error}{HTML}{fee2e2}

%* last line of the preamble
\usepackage{subfiles}

\begin{document}

%* title pages and abstract
\titul{Programovací jazyk Rust jako náhrada za C++}
{Adam Hrnčárek}{Mgr. Marek Blaha}{Brno 2023}

\podekovani{Chtěl bych poděkovat Rust komunitě za skvělé učební materiály,
    díky kterým Rustu rozumím. Také bych chtěl poděkovat YouTuberovi TheCherno,
    jehož C++ série mě naučila všechny koncepty uvedené v~této práci.
    Na závěr bych rád poděkoval svému profesoru informatiky a vedoucímu práce
    Marku Blahovi.
}

\prohlasenimuz{V~Brně dne \today}

\abstract{This thesis introduces the Rust programming language and some of it's advantages
    compared to C++. The thesis contains examples of Rust syntax and assumes no previous
    knowledge of the language.
}{Specifically, this thesis focues on major changes in the style of programming (working
    with strings and errors, object oriented programming).
}
\abstrakt{Tato závěrečná práce představuje programovací jazyk Rust a některé jeho výhody
    oproti jazyku C++. Práce obsahuje ukázky syntaxu a nepředpokládá žádnou předchozí
    znalost Rustu.
}{Specificky se práce soustředí na zásadnější změny ve~stylu programování (práce
    s~řetězci, zpracování chyb, objektově orientované programování).
}

\klslova{Programovací Jazyk Rust, C++, porovnání jazyků, objektově orientované
    programování, práce s~řetězci, softwarová spolehlivost a bezpečnost
}
\keywords{The Rust Programming Language, C++, language comparison, object oriented
    programming, working with strings, software reliability and safety
}

%* tables of contents
\obsah{}
% -TODO: delete at the end if necessary
\listoffigures
% \listoftables

\subfile{introduction.tex}
\subfile{syntax.tex}
\subfile{oop.tex}
\subfile{strings.tex}
\subfile{safety.tex}
% \subfile{tools.tex}
\subfile{conclusion.tex}

\begin{literatura}
    % V Lora nejsou kapitálky, takže je nutné je vložit ručně!
    \citace{cppreference}{C++ reference}{\nazev{C++ reference} [online].
    [cit. 2023-06-07]. Dostupné z~WWW: <\url{https://en.cppreference.com/w/cpp}>}
    \citace{docsrs}{Docs.rs}{\nazev{Docs.rs} [online].
    [cit. 2023-06-08]. Dostupné z~WWW: <\url{https://docs.rs/}>}
    \citace{byexample}{Rust By Example}{\nazev{Rust By Example} [online].
    [cit. 2023-06-07]. Dostupné z~WWW: <\url{https://doc.rust-lang.org/rust-by-example/}>}
    \citace{ruststd}{Rust STD}{\nazev{The Rust Standard Library} [online].
    [cit. 2023-06-07]. Dostupné z~WWW: <\url{https://doc.rust-lang.org/std/}>}
    % \citace{cargobook}{The Cargo Book}{\nazev{The Cargo Book} [online].
    % [cit. 2023-06-07]. Dostupné z~WWW: <\url{https://doc.rust-lang.org/cargo/}>}
    \citace{thebook}{The Book}{\autor{KLABNIK, Steve} a \autor{Carol NICHOLS}.
        \nazev{The Rust programming language} [online]. 2nd\,Edition.
        San Francisco: No Starch Press, 2023 [cit. 2023-06-06]. ISBN~9781718503106.
        Dostupné z~WWW: <\url{https://doc.rust-lang.org/book/}>}
    \citace{reference}{The Rust Reference}{\nazev{The Rust Reference} [online].
    [cit. 2023-06-07]. Dostupné z~WWW: <\url{https://doc.rust-lang.org/reference/}>}

\end{literatura}

\end{document}
