\documentclass[main.tex]{subfiles}

\begin{document}
\kapitola{Úvod a~cíl práce}

\sekce{Úvod do problematiky}

Při vývoji aplikací je nutné si vybrat programovací jazyk, který budeme na vývoj
používat. Pokud nám záleží na tom, aby byl výsledný produkt co nejrychlejší,
je historicky naše volba jasná: C++. Jazyk C++ je ale hodně komplikovaný a nemá
jednoduchá řešení pro některé problémy moderních vývojářů.

Jedním z~těchto problémů je bezpečnost při paralelním programování. Právě proto
vznikl Rust. Tento poměrně nový jazyk s~sebou přináší velké množství inovací, přesto
se ale jedná o~jazyk s~rychlostí podobnou C++.

V~posledních letech začíná Rust vytlačovat C++ v~mnoha oblastech vývoje, především
díky svým bezpečnostním zásadám.

\sekce{Cíl práce}

Cílem této práce je posoudit výhody a nevýhody Rustu oproti C++\footnote{Nikoliv
    tedy oproti jazykům, jako je například TypeScript. Ty sice Rust také nahrazuje,
    ale z~důvodu své rychlosti, ne jiných výhod.
}. Porovnání bude čerpat z~obecných informací o~obou jazycích a také z~praktické části,
ve~které bude vytvořeno několik programů se~stejnou funkcionalitou v~obou jazycích.
Práce se bude věnovat hlavně:
\begin{itemize}
    \item syntaxi a obvyklým návrhovým vzorům
    \item práci s~errory
    \item makrovému systému
    \item objektově orientovanému programování
    \item bezpečnosti programů
    \item práce se stringy
    \item vývojářským nástrojům
\end{itemize}

% TODO: zdroj podobnosti rychlosti
Dále bude v~práci obsažen stručný úvod do programovacího jazyka Rust. Cílem práce
\emph{není} porovnávání rychlosti C++ a Rustu, jelikož jsou zhruba stejně rychlé.

\sekce{Předpoklady}

Je předpokládána znalost jazyka C++, základních konceptů programování pro více vláken,
OOP, základů softwarového testování.

\end{document}
