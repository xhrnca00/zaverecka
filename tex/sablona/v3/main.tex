\documentclass[twoside,12pt]{article}
% pro tisk po jedné straně papíru je potřebné odstranit volbu twoside
% !TEX program = xelatex

%* packages
\usepackage{xdipp}
\usepackage[
     cachedir=./minted/,
     outputdir=../../TempTeX/sablona/v3, % FIXME: change path
]{minted}
\usepackage{xcolor}

%* minted options
\setminted{autogobble,
     % linenos, % line numbers
     % xleftmargin=25pt, % space for line numbers to fit inside
     mathescape,
     texcomments,
     obeytabs,
     style=tango, % algol_nu
     bgcolor=m-bg,
     highlightcolor=m-hg-default,
     tabsize=4,
}
\newminted[rustcode]{rust}{}
\newmint[rust]{rust}{}
\newmintedfile[rustfile]{rust}{stripall}
\newminted[cppcode]{cpp}{}
\newmint[cpp]{cpp}{}
\newmintedfile[cppfile]{cpp}{stripall}

%* patch minted
\makeatletter
% replace \medskip before and after the box with nothing, i.e., remove it
\patchcmd{\minted@colorbg}{\medskip}{}{}{}
\patchcmd{\endminted@colorbg}{\medskip}{}{}{}
\makeatother

%* setting options
% \cestina % implicitní
% \slovencina
% \english
% \pismo{LModern}
\pismo{Lora} % nic (=LModern), Academica, Baskerville, Bookman, Cambria, Comenia, Constantia, Palatino, Times
% \dvafonty % implicitně je \jedenfont
% \technika
% \beletrie % implicitní
\popiskyzkr
% \popisky % implicitní
\pagestyle{headings} % implicitní
\cislovat{2}
% \bakalarska % implicitní
\zaverecna
% \diplomova
% \disertacni
\brokenpenalty 10000

%* debugging
% \usepackage{showframe}
\overfullrule=4cm % shows slight overfulls as big black rectangles

%* hyphenation
\hyphenation{troj-úhel-ník}

%* colors
% courtesy of Tailwind CSS (*-100)
\definecolor{m-bg}{HTML}{f3f4f6}
\definecolor{m-hg-default}{HTML}{ecfccb}
\definecolor{m-hg-info}{HTML}{dbeafe}
\definecolor{m-hg-alert}{HTML}{fef3c7}
\definecolor{m-hg-error}{HTML}{fee2e2}

%* last line of the preamble
\usepackage{subfiles}

\begin{document}

%* title pages and abstract
\titul{Programovací jazyk Rust jako náhrada za jazyk C++}
{Adam Hrnčárek}{Mgr. Marek Blaha}{Brno 2023}

\podekovani{Děkuji mamince, že mi vařila :)}

\prohlasenimuz{V~Brně dne \today}

\abstract{Rybička, J. The \XeLaTeX{} style xdipp.sty for theses preparation.
     Example of Bachelor
     thesis. Brno, 2013.}{The typesetting style was developed and
     presented on bachelor thesis example.}

\abstrakt{Rybička, J. Styl xdipp.sty pro sazbu závěrečných prací v~systému
     \XeLaTeX. Bakalářská práce (příklad). Brno, 2013.}{Je zde popsán sazební styl
     přizpůsobený pro systém \XeLaTeX{} a~příklad jeho aplikace na bakalářskou práci.
     Sazební styl lze použít na běžné závěrečné
     práce (bakalářské, diplomové a~disertační). Do stylu verze 2.0 byly nově zavedeny
     různé volby umožňující větší flexibilitu výstupu a~využití možností fontového
     aparátu.}

\klslova{\XeLaTeX, závěrečná práce, sázecí styl}
\keywords{\XeLaTeX, thesis, typesetting style}

%* tables of contents
\obsah
\listoffigures
\listoftables

\kapitola{Úvod a~cíl práce}
\subfile{code}

\sekce{Úvod do problematiky}
Závěrečné práce patří mezi dokumenty, kterými absolvent určitého studia
prokazuje svou schopnost odborné a~případně i~vědecké práce.

Vlastní vypracování díla předpokládá využití některé technologie pro
zpracování textů. K~tomu může sloužit především typograficky i~technicky
propracovaný systém \XeLaTeX{}, v~němž lze dosáhnout s~poměrně malým úsilím
maximálně kvalitního výsledku.

Pomůckou pro sazbu dokumentu je využití vhodného sázecího stylu. K~účelu sazby
závěrečných prací byl před několika lety vyvinut styl \texttt{dipp.sty} pro
systém \LaTeXe. Od té doby se situace v~možnostech počítačové sazby pomocí
systémů postavených na bázi \TeX u změnila vyvinutím nadstavby \XeLaTeX, která
významně rozšiřuje možnosti (mimo jiné) při použití všech typů písem dostupných
na daném počítači a~zpracováním vstupního textu v~kódování UTF-8. Zjednodušila
se tak řada případů sazby speciálních znaků a~významně se zvětšily možnosti
vhodné typografické úpravy volbou široké palety písem.

Po zkušenostech s~provozem dosavadního stylu \texttt{dipp.sty} bylo přistoupeno
k~revizi a~doplnění v~souvislosti se systémem \XeLaTeX. Výsledkem je inovovaný
styl \texttt{xdipp.sty}, jehož popis je předmětem tohoto textu.

\sekce{Cíl práce}
Cílem práce je pomoci autorům závěrečných (bakalářských, diplomových
a~disertačních) prací při úpravě textu přepracováním sázecího stylu obsahujícího
nejpotřebnější prvky a~reagujícího na nové možnosti sazby v~systému \XeLaTeX.

\kapitola{Přehled literatury}
Při řešení stylu byly využity zejména zdroje typografických a~technických
informací.

Podíváme-li se na prameny, které se zabývají obsahem dokumentu, najdeme
zejména normu pro úpravu obsahu disertačních prací \cite{csn7144}, z~níž
můžeme vhodným zjednodušením dospět i~k~obsahovým doporučením pro bakalářskou
či diplomovou práci.

Bibliografické citace a~odkazy na ně řeší příslušná norma \cite{csniso690}.
Norma definuje zásady, z~nichž lze využít zejména zásadu jednotnosti. V~informativní
příloze A~této normy jsou uvedeny tři způsoby odkazování na zdroje, z~nichž
použijeme harvardskou metodu. Ta je vzhledem k~účelu a~použité technologii
pravděpodobně optimální.

Formální stránkou dokumentů se zabývá několik zásadních zdrojů. Především je potřeba
problém formální úpravy chápat opět jako složení více rozdílných prvků, které
musí navzájem tvořit vyvážený a~kompaktní celek. Jedná se o~tyto prvky:
\begin{itemize}
     \item písmo a~použité symboly,
     \item vyjádření odstavců, výčtů, poznámek,
     \item vložení neodstavcových objektů (tabulek, obrázků, matematických výrazů),
     \item úprava stránek.
\end{itemize}

Základním zdrojem informací zabývajícím se úpravou dokumentů je bezesporu
norma ČSN 01~6910 -- Úprava dokumentů zpracovaných textovými procesory z~roku
2014. Řada případů je obsažena i~v~pravidlech českého pravopisu, jejichž
snadno použitelná podoba je k~dispozici online \cite{pravidla}.

Pro vytvoření kvalitního výstupu je nezbytné znát alespoň základní typografická
pravidla, která se vyvíjejí již více než 500 let a~uplatňují se v~každém
dokumentu s~knižním písmem. Vhodnými zdroji jsou například učebnice a~manuály
\cite{pop, beran, latzac}.

Sazbu tabulek souhrnně zpracovává diplomová práce P.\,Talandové
\cite{TalDipl}. Přestože se nejedná o~primární pramen, je možné odsud převzít
jak typografickou, tak i~technickou část řešení problému.

Sazba matematická a~chemická je pro dřívější technologii sazby dobře popsána
v~Nohelově učebnici \cite{Nohel}. Některá zde uvedená ustanovení pro způsob
sazby matematických výrazů platí doposud. Pro matematické značky, jejich
význam a~způsob zápisu lze s~výhodou využít platnou normu ČSN ISO~80000 a~její
odpovídající části \cite{csn80000, csn80000m}.

\kapitola{Návrh řešení}

\sekce{Sazební styl a~jeho použití}
Cílem sazby je získání optimálního výsledku z~hlediska užitných vlastností --
zejména čitelnosti a~estetické kvality. Dosažení tohoto cíle je v~různých
počítačových systémech různě složité. Systém \TeX{} s~nadstavbou \LaTeX, který
byl již od prvopočátku tvořen pro nejvyšší možnou kvalitu výstupu, je tedy
zcela optimální. Určitou překážkou je jistá složitost, jež dělá potíže zejména
začátečníkům, ale po potřebné průpravě (která není delší než u~jiných systémů)
a~při existenci podpůrného stylu lze připravit velmi rychle text, jehož výstup
je zcela precizní.

Cílem této kapitoly je popsat připravený sazební styl systému \XeLaTeX{}
a~postup jeho použití. Systém \XeLaTeX{} představuje významné rozšíření možností
systému \LaTeX. Jde zejména o~možnost využití moderních fontů a~všech jejich
znaků. Základním kódováním vstupu je UTF-8, čímž lze snadno zapisovat všechny
národní znaky různých abeced i~další speciální znaky (pomlčky, uvozovky,
matematické značky atd.).

Dostupnost sázecího systému je výrazně zvýšena webovou aplikací \TeX onWeb
\cite{texonweb}. Potenciální uživatel tedy nepotřebuje žádné speciální vybavení,
postačí pouze webový prohlížeč a~čtečka souborů formátu PDF.

\podsekce{Obecné vlastnosti systému}
Vstupem do systému je textový soubor (dokument), který kromě vlastního textu
obsahuje i~\textit{příkazy pro sazbu}. Tyto příkazy se zapisují naprosto
stejně jako obyčejný text. Vstupní soubor lze tedy vytvořit jakýmkoliv
editorem v~libovolném operačním systému. Uživatel si může zvolit prostředek,
na který je  zvyklý a~s~nímž se mu bude dobře pracovat.

Příkazy jsou předdefinovány, ale uživatel si může tvořit své vlastní, čímž si
přizpůsobuje systém svým potřebám. Předdefinovaných příkazů jsou řádově
stovky, připojením různých stylů je možné toto množství ještě dále zvyšovat.
Pro běžnou práci je však potřebné znát pouze určité poměrně malé množství
nejfrekventovanějších příkazů.

V~tomto popisu se zmíníme o~základních vlastnostech v souvislosti s použitím
sázecího stylu. Pokud čtenář v systému \LaTeX{} dosud nepracoval, silně
doporučujeme získat podrobnější informace například v~učebnici \cite{latzac}.

%----------------- příkazy pro sazbu příkazů
\def\bsl{\char92}
\def\lsv{\char123}
\def\rsv{\char125}
%----------------- konec mezidefinic

\sekce{Příprava vlastního textu}
Nejprve se seznámíme se způsobem sazby hladkého textu, který budeme potřebovat
všude. Sazba vybraných speciálních znaků je uvedena souhrnně v~tab.~\ref{speczn}.
Uvedeme-li v~kterémkoliv místě textu znak procento (\texttt{\%}), veškerý text
až do konce řádku nebude zpracováván, tvoří poznámku.

\tabulka{Sazba speciálních znaků v~hladkém textu}

\label{speczn}
\def\arraystretch{1.2}
\footnotesize
\tabcolsep 2.5pt
\begin{tabular}{|p{.16\textwidth}|p{.48\textwidth}|l|l|} \hline
     \textbf{Znak}                                           & \textbf{Zápis v~textu}                                            & \textbf{Příklad}            & {\bfseries
     Vysázeno}                                                                                                                                                              \\ \hline\hline
     Mezera                                                  & Stisk mezerníku (i~několikanásobný)                               &                             &            \\ \hline
     Nezlomitelná\newline mezera                             & \texttt{\char126} (ručně nebo spuštěním
     programu pro \newline automatické vložení za předložky) &
     \texttt{u\char126okna}                                  & u~okna                                                                                                       \\ \hline
     Zúžená mezera 1/4 em                                    & \texttt{\bsl,}                                                    & \texttt{P.\bsl,Král}        & P.\,Král   \\ \hline
     Zúžená mezera 1/6 em                                    & \texttt{\bsl;} -- používá se k~mezerování výpustky                &
     \texttt{přišel\bsl;\bsl dots}                           & přišel\;\dots                                                                                                \\ \hline
     Široká mezera                                           & \texttt{\bsl quad} (čtverčík -- 1 em) \texttt{\bsl qquad} (2
     em)                                                     & \texttt{$a=b$\bsl quad $b=c$}                                     & $a=b$\quad $b=c$                         \\ \hline
     Odstavec                                                & Vynechaný řádek (i~vícenásobně)                                   &                             &            \\ \hline
     Tři tečky                                               & \texttt{...} nebo \texttt{\bsl dots}                              & \texttt{zanikl\bsl;\bsl
     dots}                                                   & zanikl\;\dots                                                                                                \\
     \hline
     Spojovník                                               & \texttt{-} (přímo z~klávesnice)                                   & \texttt{bude-li}            & bude-li    \\ \hline
     Spojovník                                               & \texttt{\bsl spoj} (tento spojovník se při řádkovém zlomu\newline
     přetahuje na začátek následujícího řádku)               & \texttt{bude\bsl spoj\lsv\rsv li}                                 &
     \polet l{bude-                                                                                                                                                         \\-li} \\ \hline
     Pomlčka                                                 & \texttt{-{}-} (půlčtverčíková) \texttt{-{}-{}-} (čtverčíková)     &
     \texttt{6-{}-12}                                        & 6--12                                                                                                        \\ \hline
     Pomlčka \newline (rozsahová)                            & \texttt{\bsl az} (tato pomlčka se při řádkovém
     \newline zlomu nahradí slovem \uv{až})                  & \texttt{6\bsl az 12}                                              & 6--12, \polet
     r{nebo 6                                                                                                                                                               \\až 12} \\ \hline
     Znak minus                                              & \texttt{\char36-\char36} (v~matematickém režimu)                  &
     \texttt{\char36-10\char36}                              & $-10$                                                                                                        \\ \hline
     Znak násobení                                           & \texttt{\char36\bsl times\char36} nebo přímo z~klávesnice ×       & \texttt{\char36 2\bsl
     times 3\char36\bsl,mm}                                  & $2\times 3$\,mm                                                                                              \\ \hline
     Stupeň                                                  & \texttt{\char36\char94\bsl circ\char36}                           &
     \texttt{5\bsl,\char36\char94\bsl circ\char36C}          & 5\,$^{\circ}$C                                                                                               \\ \hline
     Paragraf                                                & (jen s~číslem) \texttt{\bsl S} nebo přímo z~klávesnice §          & \texttt{\bsl S\bsl,36}      & \S\,36     \\ \hline
     Značky \newline \#, \$ a~\&                             & \texttt{\bsl\#}, \texttt{\bsl\char36},
     \texttt{\bsl\&}                                         &                                                                   &                                          \\ \hline
     Závorky \{ a~\}                                         & \texttt{\bsl\lsv}, \texttt{\bsl\rsv}                              &                             &            \\ \hline
     Značky $<$ a~$>$                                        & \texttt{\char36\char60\char36}, \texttt{\char36\char62\char36}
                                                             & \texttt{\char36a\char62 b\char36}                                 & $a>b$                                    \\ \hline
     Procento                                                & \texttt{\bsl\char37}                                              & \texttt{10\bsl,\bsl\char37} & 10\,\%     \\ \hline
     Uvozovky                                                & \texttt{\bsl uv\lsv}text\texttt{\rsv}                             & \texttt{\bsl uv\lsv
     Něco\rsv}                                               & \uv{Něco}                                                                                                    \\ \hline
     Uvozovky \newline (úhlové)                              & \texttt{\bsl uvv\lsv}text\texttt{\rsv}                            &
     \texttt{\bsl uvv\lsv Něco\rsv}                          & \uvv{Něco}                                                                                                   \\ \hline
\end{tabular}
\endtab

V~textu lze používat příkazy pro vyznačení -- základní vyznačení
\verb.\emph{text}., pro důležité pojmy pak tučného řezu příkazem
\verb.\textbf{text}., kapitálky jsou dostupné příkazem \verb.\textsc{text}..

\sekce{Příkazy stylu}
\podsekce{Základní tvar dokumentu}

\begin{verbatim}
     \documentclass[twoside, 12pt]{article}
     \usepackage{xdipp}
         ... nastavení parametrů
     \begin{document}
         ... vlastní text
     \end{document}
\end{verbatim}

\podsekce{Všeobecná nastavení}

V~uvedeném základním tvaru jsou implicitně nastaveny parametry sazby souhrnně
uvedené v~tab.\,\ref{parsazby}.

\tabulka{Implicitní parametry sazby závěrečné práce}
\label{parsazby}

\begin{tabular}{|l|l|}\hline
     \bfseries Parametr               & \bfseries Hodnota      \\\hline
     Sazební zrcadlo                  & $150\times 220$\,mm    \\
     Uspořádání dokumentu             & dvoustranný dokument   \\
     Stránkový design                 & běžná záhlaví s~linkou \\
     Typ práce                        & bakalářská             \\
     Jazyk dokumentu                  & čeština                \\
     Typ základního písma             & Latin Modern           \\
     Kombinace typů písma v~dokumentu & jeden typ              \\
     Stupeň základního písma          & 12\,pt                 \\
     Řádkový proklad základního písma & 2\,pt                  \\
     Odstavcová zarážka               & 24\,pt                 \\
     Vertikální odstavcové odsazení   & 0\,pt                  \\
     Názvy obrázků a~tabulek          & nezkrácené             \\
     Úroveň číslování nadpisů         & 3                      \\\hline
\end{tabular}
\endtab

\podsekce{Změny implicitních parametrů}

V~uvedeném základním tvaru můžeme nařídit změny sazby následujícími
příkazy:

\begin{itemize}
     \item Jednostrannou variantu lze nařídit smazáním
           volitelného parametru \texttt{twoside} u~prvního příkazu celého dokumentu: \\
           \verb!     \documentclass[12pt]{article}!

     \item Stránkový design lze změnit na obyčejný, kdy nejsou sázena stránková
           záhlaví, ale pouze stránkové paty, v~nichž je na vnější straně pouze číslo
           stránky. Příkaz: \\
           \verb!     \pagestyle{plain}!

     \item Typ práce -- ze základního nastavení na bakalářskou práci lze provést změnu
           příkazem:\\
           \verb.     \diplomova. -- pro diplomové práce\\
           \verb.     \disertacni. -- pro disertační práce

     \item Jazyk dokumentu -- jeho nastavení určuje výpis jazykově závislých textů
           (titulní list, prohlášení, obsah apod.), správné dělení slov a~použití některých
           dalších prvků. Jsou připraveny modifikace pro slovenštinu a~angličtinu:\\
           \verb!     \slovencina!\\
           nebo\\
           \verb!     \english!

     \item Typ písma dokumentu -- pro sazbu jiným písmem než Latin Modern je možné
           využít připravené sady. Příkaz:\\
           \verb!     \pismo{!{\itshape název}\verb!}!\\
           Jako {\itshape název} lze použít některou ze sad:
           Academica, Baskerville, Bookman, Cambria, Comenia, Constantia, Palatino, Times. U~každé sady je
           vyřešen základní typ a~jeho použití ve všech částech dokumentu včetně vybraných
           matematických symbolů a~případných strojopisných částí.

     \item Kombinace typů písma v~dokumentu -- základní tvar předpokládá, že celý
           dokument je vysázen jedním typem písma. Někteří uživatelé preferují doplnit
           základní písmo druhým (bezserifovým) písmem na nadpisy apod. U~každé sady je
           tedy definována i~kompatibilní bezserifová varianta. Její aktivace se provede
           příkazem \\
           \verb.      \dvafonty.

     \item Základní vzhled odstavců -- pro běžný text se často používá odlišení
           odstavců zarážkami, mezi odstavci je pak nulová mezera. Chcete-li změnit tento
           tvar na odlišení odstavců svislými mezerami místo zarážek (což se používá
           v~technických a~odborných textech s~menším množstvím souvislého textu a~mnoha
           tabulkami, obrázky a~dalšími prvky), lze to nařídit příkazem\\
           \verb.      \technika.

     \item Popisky obrázků a~tabulek -- vypisují se automaticky podle nastaveného
           jazyka, a~to implicitně v~nezkrácené podobě (Obrázek/Obrázok/Figure). Chcete-li
           tyto texty vypisovat zkráceně, použijte příkaz\\
           \verb.      \popiskyzkr.\\
           a~bude se vypisovat Obr./Obr./Fig.

     \item Úroveň číslování nadpisů je nastavena na 3, číslují se tedy kapitoly, sekce
           i~podsekce, zároveň jsou tyto úrovně zařazeny do obsahu. Chcete-li číslovat
           a~zařazovat do obsahu méně úrovní, použijte příkaz:\\
           \verb.      \cislovat{2}. nebo \verb. \cislovat{1}.

\end{itemize}

\podsekce{Úvodní stránky}
Pro standardní stránky závěrečné práce jsou připraveny následující příkazy:

\begin{itemize}
     \item Titulní stránka -- příkaz \\
           \verb!     \titul{název práce}{autor}{vedoucí práce}{místo a~rok}! \\
           Vytvoří se titulní stránka s~předepsanými údaji.
           Pokud je potřeba změnit název školy a~fakulty, je možné uvést makro
           \verb!\skola{text}!, resp. \verb!\fakulta{text}!. Tato makra uvedeme
           před makrem \verb!\titul!. Implicitně je vypisována Provozně ekonomická fakulta
           a~Mendelova univerzita v~Brně, tyto texty podléhají nastavení jazyka práce.
     \item Poděkování -- příkaz \\
           \verb!     \podekovani{libovolný text}!\\
           Vytvoří se nová stránka, v~jejíž dolní části je
           formátován text uvedený v~parametru příkazu.
     \item Prohlášení podle směrnice rektora -- má dvě varianty: pro ženu a~pro muže\\
           \verb!     \prohlasenizena{místo a~datum}!\\
           \verb!     \prohlasenimuz{místo a~datum}!\\
           Vytvoří se nová stránka, v~dolní části
           je kompletní text prohlášení podle vyhlášky s~automatickým dosazením
           názvu práce, pod textem je vlevo místo a~datum z~parametru makra a~vpravo
           tečky pro podpis.
     \item Abstrakty -- příkazy \\
           \verb!     \abstract{úvodní údaje}{text v~angličtině}!\\
           a\\
           \verb!     \abstrakt{úvodní údaje}{text v~češtině nebo slovenštině}!\\
           Na pořadí uvedení těchto příkazů nezáleží, budou vypisovány v~pořadí, které je
           dáno nastaveným jazykem práce. V~pracích psaných v~češtině a~slovenštině se
           nejprve vypisuje anglický abstrakt, v~pracích psaných v~angličtině se nejprve
           vypisuje abstrakt v~češtině (slovenštině).

     \item Klíčová slova -- příkazy\\
           \verb.     \klslova{seznam klíčových slov v~češtině/slovenštině}.\\
           \verb.     \keywords{seznam klíčových slov v~angličtině}.

     \item Obsah -- příkaz \\
           \verb.     \obsah. \\
           Příkaz vytvoří novou stránku, na níž bude vysázen obsah složený z~údajů
           titulků. Do obsahu budou implicitně zahrnuty jen číslované titulky. Úroveň
           číslování (tj.\ jak důležitý titulek bude zahrnut do obsahu) se řídí příkazem
           \\ \verb!\cislovat{úroveň}! \\
           kde \texttt{úroveň} je číslice 1, 2 nebo 3. Hodnota 1 -- číslují se jen
           kapitoly, 2 -- číslují se kapitoly a~sekce, 3 -- číslují se kapitoly, sekce
           a~podsekce.
     \item Seznam obrázků nebo tabulek -- příkazy \verb.\listoffigures., resp.
           \verb.\listoftables.. Tyto příkazy lze uvést do místa v~textu, kam chcete
           seznam vysázet. Logicky patří za \verb.\obsah., jako je tomu v~tomto textu.
\end{itemize}

\podsekce{Oddíly textu}
Text práce je členěn do kapitol, sekcí a~podsekcí.

Každá kapitola začíná vždy na nové stránce. Její hlavička se zanáší do obsahu
a~do záhlaví stránek. Novou kapitolu nařídíme příkazem
\verb.\kapitola{titulek}.. Podobně nařídíme novou sekci příkazem
\verb!\sekce{titulek}!, resp.\,podsekci příkazem \verb!\podsekce{titulek}!.

Výjimkou je seznam literatury -- tomuto úseku bude věnován zvláštní oddíl.

Práce může obsahovat přílohy. Místo, kde začínají přílohy, se označí příkazem
\verb.\prilohy. nebo \verb.prilohy*.. Příkaz bez hvězdičky vytvoří v~místě
uvedení samostatnou stránku s~nadpisem \uv{Přílohy}. Příkaz s~hvězdičkou tuto
stránku nevytváří. V~obou případech se však nařídí změna číslování --
jednotlivé přílohy jsou označeny velkými písmeny.

Každá příloha je uvozena příkazem \verb.\priloha{titulek}. Její název je
zanesen do obsahu a~do běžných záhlaví (pokud jsou nařízena).

\podsekce{Seznam literatury}
Soupis citací je jednou z~nejdůležitějších částí práce, která vyžaduje
pečlivost a~přesnost. Proto je také ve stylu podpořena řadou nástrojů.

Samotný seznam literatury je nařízen příkazem pro prostředí
\begin{verbatim}
     \begin{literatura}
     \citace...
     \citace...
        ...
     \end{literatura}
\end{verbatim}
U~prostředí se automaticky vysadí titulek úrovně kapitoly s~názvem
\uv{Literatura} (v~závislosti na nastaveném jazyce také \uv{Literatúra} nebo
\uv{References}. Pokud by bylo potřeba tento titulek změnit, stačí napsat
\textit{před} začátek prostředí \texttt{literatura} tento příkaz: \\
\verb!     \def\refname{cokoliv}! \\
a~titulek bude změněn na \texttt{cokoliv}.

Jednotlivé položky literatury jsou řešeny příkazem se třemi parametry: \\
\verb:     \citace{návěští}{tvar odkazu}{text citace}: \\
V~prvním parametru je libovolný řetězec znaků, který představuje symbolické
pojmenování daného zdroje, na něž se lze odvolávat v~textu (provést
automatizovaný odkaz). Tvar odkazu, který se má v~textu objevit, je obsahem
druhého parametru. Ve třetím parametru jsou pak jednotlivé údaje, jak
odpovídají druhu citačního záznamu podle normy. Jméno autora se zde
sází kapitálkami -- k~tomu slouží příkaz \verb.\autor{jméno}., název zdroje se
sází kurzívou -- k~tomu slouží příkaz \verb.\nazev{název}.. Ostatní údaje se
uvádějí obyčejným řezem.

Pro automatizované odkazy v~textu slouží příkaz \verb.\cite.. Tento příkaz má
jeden volitelný a~jeden povinný parametr: \\
\verb.     \cite[dodatek]{návěští}.\\
Povinný parametr udává návěští příslušné citace (tj.\,řetězec, který je
u~některého příkazu \verb.\citace. v~prvním parametru). Volitelný parametr může
obsahovat některé upřesňující údaje, například číslo stránky daného odkazu.
Příkaz \verb.\cite. může obsahovat v~povinném parametru i~seznam návěští
oddělených čárkami. Ve vysazeném odkazu se pak jednotlivé prameny oddělují
automaticky středníky.

Příklad: předpokládejme, že v~seznamu literatury se objevuje položka:
\begin{verbatim}
     \citace{oceli18}{Novák, 1991}{\autor{Novák, J.} a~kol.
     \nazev{Konstrukční vlastnosti ocelí třídy 18}. Praha:
     SNTL, 1991. 439~s. ISBN 80-8432-289-9.}
\end{verbatim}
Pak se příkazem \verb!\cite[s.\,52]{oceli18}! uvedeným v~textu objeví odkaz ve
tvaru (Novák, 1991, s.\,52). Máme tedy jistotu, že odkaz bude ve všech místech
stejný, protože se odkazujeme pouze na návěští, jehož správná podoba se při
zpracování textu automaticky kontroluje.

\podsekce{Křížové odkazy}
Na jakékoliv místo v~textu (číslo kapitoly, sekce a~podsekce), obrázek,
tabulku nebo číslovaný výraz se lze odvolat pomocí tzv. \textit{křížového
     odkazu}. Základní princip spočívá v~symbolickém pojmenování daného místa
pomocí příkazu \verb.\label{návěští}. Na toto návěští se pak odvoláváme
příkazem \verb.\ref{návěští}. (dostaneme příslušné číslo) nebo příkazem
\verb!\pageref{návěští}! (dostaneme číslo stránky, na níž se nachází značka
\uv{položená} příkazem \verb.\label.).

\podsekce{Obrázky a~tabulky}
Obrázky a~tabulky se mohou umístit přímo do textu nebo do tzv.\,plovoucích
prostředí, která umožňují jejich automatické vložení do vhodného místa
vysázeného tvaru.

Je-li obrázek ve vektorové podobě, lze s~ním provádět řadu operací, které
nemají vliv na kvalitu zobrazení (zejména zvětšování a~zmenšování, otáčení
apod.). V~dnešní době není problém vytvořit vektorový obrázek trasováním
rastrového, získaného například skenováním nebo podobným procesem. Problémem
zůstávají pouze fotografie, které vzhledem k~použité tiskové technologii
potřebují poměrně vysokou hustotu, jsou rastrové a~vyžadují obvykle
předzpracování v~některém výkonném rastrovém obrazovém editoru. To ostatně
platí pro jakýkoliv publikační systém.

Pro vložení obrázku do sazby jsou vhodné tyto formáty:
\begin{itemize}
     \item JPG -- rastrový obraz vhodný zejména pro fotografie. Vyniká silnou
           kompresí, zabírá tedy relativně málo prostoru. Velikost výsledného obrazu
           je dána také požadovanou kvalitou. Pro tiskové výstupy je potřebné obrazy
           připravit tak, aby jejich barevná hloubka a~hustota odpovídaly kvalitě tisku.
           Pro tisk se doporučuje hustota min.\,300\,dpi. Barevná hloubka je u~barevné
           fotografie typicky v~pravých barvách.
           Je ovšem velmi vhodné provést zkušební tisk takto připravených obrázků, neboť jen
           na papíře lze odpovědně posoudit, jak kvalitní obrázek je.
     \item PNG -- rastrový obraz vhodný pro monochromatické obrazy nebo nižší
           barevné hloubky. Nepoužívá ztrátovou kompresi, opět je však velmi vhodné
           vyzkoušet tiskem na papír, s~jakou hustotou by měly být obrázky připraveny,
           aby byl výsledek dostatečně kvalitní.
     \item PDF -- vektorový obraz. Pozor ovšem na obrazy, které jsou sice uloženy
           ve formátu PDF, ale vnitřně obsahují pouze rastrový obraz, například výstupy
           ze skenerů a~různých jiných programů.
\end{itemize}
Pro vložení do dokumentu jsou v~sazebním stylu předdefinovány tři příkazy:
%\draft
\begin{itemize}
     \item Příkaz \verb!\vlozobr{jméno souboru}{měřítko}!, jeho prvním parametrem je
           jméno souboru s~obrázkem, druhým parametrem pak koeficient, kterým se upraví
           výsledná velikost (hodnota 1 = beze změny, hodnota $<1$ = zmenšení, hodnota
           $>1$ = zvětšení). Tento koeficient je zapisován jako číslo v~anglickém pravopisu,
           tedy s~desetinnou tečkou. Příklad:\\
           \verb!\vlozobr{motyl.jpg}{0.75}! \\
           příkaz vloží fotografii ze souboru {\ttfamily motyl.jpg} zmenšenou na 75\,\%
           původní velikosti.

     \item Pokud je potřebné obrázky přizpůsobit na určitý rozměr, lze použít
           příkaz\\
           \verb!\vlozobrbox{jméno souboru}{rozměr-x}{rozměr-y}!\\
           kde {\ttfamily rozměr-x} nebo {\ttfamily rozměr-y} je požadovaný rozměr
           v~daném směru, uvádí se jako číslo s~délkovou jednotkou\footnote{Délkovými
                jednotkami mohou být cm -- centimetry, mm -- milimetry, pt -- anglické
                typografické body, bp -- tzv.\,\uv{velké} (též monotypové) anglické typografické body,
                pc -- picas, dd -- evropské typografické body, cc -- cicera,
                in -- palce. Lze použít také relativní jednotky: em -- čtverčík, ex -- půlčtverčík.}.
           Každý rozměr může být také zadán pomocí délkového
           registru.\footnote{Délkovým registrem může být předdefinovaný nebo vlastní
           registr, pro zadání rozměru může být registr ještě násoben libovolným
           koeficientem. Například šířka sazby je v~délkovém registru
                {\ttfamily\bsl textwidth}, můžeme tedy obrázek upravit přesně na 90\,\%
           šířky sazby
           příkazem {\ttfamily\bsl vlozobrbox\lsv schema.pdf\rsv{}\lsv{}0.9\bsl textwidth\rsv
           \lsv!\rsv}.}
           Jsou-li uvedeny oba rozměry, obrázek se může tvarově změnit.
           Pokud chceme definovat pouze jeden z~rozměrů a~druhý necháme spočítat tak,
           aby se nezměnil původní poměr stran, uvedeme místo druhého rozměru znak
           vykřičník. Například příkaz: \\
           \verb:\vlozobrbox{schema.pdf}{130mm}{!}:\\
           vloží do sazby obrázek v~souboru {\ttfamily schema.pdf}, upraví jeho
           vodorovný rozměr na 130\,mm a~svislý rozměr nastaví tak, aby původní poměr
           stran obrázku zůstal zachován.

\end{itemize}

Máme-li v~textu více obrázků a~chceme pracovat jen s~textem, je většinou
výsledný soubor zbytečně veliký (jsou v~něm vloženy obrázky) a~poněkud to
zpomaluje případné přenosy a~zobrazení. Ve stylu je k~tomuto účelu definován
příkaz {\ttfamily\bsl draft}. Když ho uvedeme kdekoliv v~textu, od tohoto
místa až do konce dokumentu se místo obrázků udělá jen prázdný obdélník,
v~němž je napsáno jméno obrázkového souboru. Vložením tohoto příkazu do
skupiny ohraničené některým prostředím nebo svorkami můžeme působnost
omezit na libovolný úsek dokumentu.

Vložíme-li obrázek do plovoucího prostředí, můžeme kromě vhodného umístění
rovněž definovat popisek obrázku. Obrázky se automaticky číslují. Plovoucí
obrázek nařídíme příkazy: \\ \verb!     \obrazek! \\\dots libovolný materiál,
nejčastěji \verb.\vlozobr. nebo \verb.\vlozobrbox. \\
\verb.     \endobr{popisek}.\\
Pokud má obrázek mít jen číslo bez popisku, vložíme místo příkazu
\verb.\endobr. příkaz \verb.\endobrbez..

Chceme-li, aby plovoucí prostředí nebylo vloženo do textu, ale zaujímalo
samostatnou stránku, použijeme místo \verb.\obrazek. příkaz \verb.\obrazekp..

Popisky obrázků se umísťují vždy pod obrázek, proto je popisek definován až
v~závěrečném příkazu. Uvnitř popisku můžeme uvést příkaz
\verb.\obrzdroj{odkaz}., v~němž uvedeme, odkud je obrázek převzat. Příkaz
automaticky vypíše řetězec \uv{Zdroj} podle aktuálního nastavení jazyka práce.
Samozřejmě zde neuvádíme nesmysl typu \verb.\obrzdroj{vlastní práce}.!

Příklady obrázků: Obrázek~\ref{prikobr1} byl vložen následujícími příkazy:
\begin{verbatim}
     \obrazek
     \vlozobrbox{susic.png}{.7\textwidth}{!}
     \endobr{Schematické znázornění hlavní funkce sušiče
            \obrzdroj{\cite[s.\,36]{atlaslok3}}}
\end{verbatim}
\obrazek
\vlozobrbox{susic.png}{0.7\textwidth}{!}
\endobrl{Schematické znázornění hlavní funkce sušiče
     \obrzdroj{Bek, 1982, s.\,36}}{prikobr1}
Obrázek~\ref{prikobr2} byl vložen těmito příkazy:
\begin{verbatim}
     \obrazek
     \vlozobr{rozvod}{0.8}
     \endobr{Schéma rozvodu ve zvětšení 0,8
          \obrzdroj{\cite[s.\,29]{atlaslok2}}}
\end{verbatim}
\obrazek
\vlozobr{rozvod}{0.8}
\endobrl{Schéma rozvodu ve zvětšení 0,8\obrzdroj{Bek, 1983, s.\,29}}{prikobr2}

Chceme-li se na obrázek odvolávat v~textu, je velmi vhodné tuto odvolávku
udělat symbolicky, nikoliv zápisem konkrétního čísla obrázku (obrázky se
mohou přečíslovat,
pokud například nějaký obrázek přidáme nebo zrušíme). K~tomu je potřebné
definovat symbolické jméno daného obrázku, na které se můžeme pak kdekoliv
v~textu odvolávat. Za tímto účelem je definován příkaz \texttt{\bsl endobrl},
který má dva parametry: prvním parametrem je popisek obrázku, druhým parametrem
je pak zvolené symbolické jméno obrázku. V~textu se pak na takový obrázek
můžeme odvolat příkazem {\ttfamily\bsl ref\lsv sjmeno\rsv}. Příklad: u~obrázku
použijeme příkaz\\
\verb!\endobrl{Schéma rozvodu v~měřítku $1:50$}{schema}!\\
v~textu se pak na obrázek odvoláme: \\\uv{{\ttfamily Detaily jsou viditelné
               na obr.\,\bsl ref\lsv schema\rsv}}.

U~tabulek je situace obdobná, popisek je však \textit{nad} tabulkou, proto se
uvádí na začátku: \\
\verb.     \tabulka{popisek}\label{odkaz}. materiál tabulky \verb.\endtab.

Pro celostránkovou tabulku lze použít příkaz \verb.\tabulkap{popisek}..

I~u~tabulek je často potřebné uvést zdroj, pokud jsou skutečně data přebírána
a~nejsou pouze vlastní prací autora. K~tomu slouží makro \verb!\tabzdroj{odkaz}!,
který umístíme těsně před ukončující příkaz \verb.\endtab..

Příklad tabulky -- následující příkazy slouží k~vysazení tabulky~\ref{zdroje}:
\begin{verbatim}
 \tabulka{Údaje o~frekvencích beta testů}

 \label{zdroje}
 \vykricnik % vykřičník nyní nahrazuje číslicovou mezeru
 \def\arraystretch{1.2}
 \begin{tabular}{|l|c|c|} \hline
 \textbf{Druh algoritmu} & \textbf{\pole c{Charakter\\testu}} &
 \textbf{Frekvence} \\ \hline
 Vyhledávání & čas. + prost. & $25\cdot 10^3$ \\
 Řazení      & čas.& $!6\cdot 10^4$ \\ \hline
 \end{tabular}
 \tabzdroj{sumarizováno podle výzkumů provedených na ČVUT
            \cite[s.\,338]{Betatesty}}
 \endtab
\end{verbatim}

\tabulka{Údaje o~frekvencích beta testů}

\label{zdroje}
\vykricnik % vykřičník nyní nahrazuje číslicovou mezeru
\def\arraystretch{1.2}
\begin{tabular}{|l|c|c|} \hline
     \textbf{Druh algoritmu} & \textbf{\pole c{Charakter                  \\testu}} &
     \textbf{Frekvence}                                                   \\ \hline
     Vyhledávání             & čas. + prost.             & $25\cdot 10^3$ \\
     Řazení                  & čas.                      & $!6\cdot 10^4$ \\ \hline
\end{tabular}
\tabzdroj{sumarizováno podle výzkumů provedených na ČVUT (Kratochvíl, 2017, s.\,338)}
\endtab

Symbolický odkaz na tabulky je rovněž možný, kdekoliv v~textu se můžeme
na tabulku odvolat opět příkazem {\ttfamily \bsl ref}. Symbolické jméno tabulky
můžeme definovat příkazem {\ttfamily \bsl label\lsv sjmeno\rsv} uvnitř
prostředí {\ttfamily \bsl tabulka\lsv popisek\rsv{} ... \bsl endtab}.

Popisky obrázků a~tabulek začínají standardně textem \uv{Obrázek},
resp.\,\uv{Tabulka}, případně zkrácenými názvy \uv{Obr.} a~\uv{Tab.} (obojí navíc
v~závislosti na nastavení aktuálního jazyka práce). Tyto texty lze změnit příkazem \\
\verb!     \def\figurename{cokoliv}   ! pro změnu názvu obrázku na cokoliv a\\
\verb!     \def\tablename{cokoliv}    ! pro změnu názvu tabulky na cokoliv.

\medskip
A~na závěr této sekce ještě jedno \uv{typografické} pravidlo: za texty
popisků tabulek a~obrázků se nepíše tečka.

\podsekce{Výčty}
Pro nečíslované a~číslované výčty jsou k~dispozici standardní prostředí
\texttt{itemize} nebo \texttt{enumerate}. Položky začínají příkazem \verb.\item..
Standardně je mezi okolním textem a~výčtem oddělující svislá mezera, rovněž
mezi jednotlivými položkami výčtu je zvětšená mezera. Mezeru mezi položkami
výčtu lze upravit nastavením registru \verb.\itemsep.. Uvedeme dva příklady
výčtů -- první bude nečíslovaný a~druhý číslovaný:

\begin{verbatim}
Ekologické iniciativy argumentují obvykle následujícími tvrzeními:
\begin{itemize}
\item Prostředí se zhoršuje vlivem spalování ropy v~automobilech.
\item Exhalacemi CO$_2$ se zvyšuje skleníkový efekt.
\item Přestože zvýšení teploty je na první pohled malé, dochází
k~tánívětšího množství arktického a~antarktického ledu, čímž se
zvyšuje hladina světového oceánu.
\end{itemize}
\end{verbatim}

{\small (po vysázení:)}

\noindent Ekologické iniciativy argumentují obvykle následujícími tvrzeními:
\begin{itemize}
     \item Prostředí se zhoršuje vlivem spalování ropy v~automobilech.
     \item Exhalacemi CO$_2$ se zvyšuje skleníkový efekt.
     \item Přestože zvýšení teploty je na první pohled malé, dochází k~tání většího
           množství arktického a~antarktického ledu, čímž se zvyšuje hladina světového
           oceánu.
\end{itemize}

Druhý příklad:

\begin{verbatim}
Pořadí, v~němž je nutné postupovat v~bakalářském semináři:
\begin{enumerate} \itemsep=0pt
\item Výběr tématu.
\item Oslovení učitele -- vedoucího práce.
\item Vytvoření záměru bakalářské práce.
\item Vypracování rešerše.
\end{enumerate}
\end{verbatim}

{\small (po vysázení:)}

\noindent Pořadí, v~němž je nutné postupovat v~bakalářském semináři:
\begin{enumerate} \itemsep 0pt
     \item Výběr tématu.
     \item Oslovení učitele -- vedoucího práce.
     \item Vytvoření záměru bakalářské práce.
     \item Vypracování rešerše.
\end{enumerate}

\podsekce{Matematické výrazy}

V~textech závěrečných prací se často vyskytují matematické výrazy, pro něž je
nejkvalitnější a~nejjednodušší systém právě \TeX{} a~všechny jeho nadstavby.
Například výraz definující parciální derivaci

\begin{equation}
     \frac{\partial f(x_1,x_2)}{\partial x_1}=\lim_{\Delta x_1\rightarrow 0}
     \frac{f(x_1+\Delta x_1, x_2) - f(x_1, x_2)}{\Delta x_1}
\end{equation}
nebo také známý výraz
\begin{equation}
     \hbox{Mezní míra substituce}=\frac{\hbox{malá změna } \Delta x_2}{\hbox{malá
               změna } \Delta x_1}
\end{equation}
kde \begin{tabular}[t]{l}
     $x_1$ je množství statku 1, \\
     $x_2$ je množství statku 2, \\
     $\Delta$ je značka pro diferenci (změnu)
\end{tabular} \\
jsou zapsatelné velmi pohodlně, čitelně a~robustně, výsledek je dokonalý.
Výrazy mohou být automaticky číslovány (jako v~této ukázce) a~na tato čísla
lze vytvořit automatické křížové odkazy příkazem {\ttfamily\bsl ref}. Stejně
dobře je možné veškeré
matematické symboly, jako třeba $\Delta, \pi, \infty$ atd.\ vkládat do
textového materiálu v~odstavcích.

\sekce{Jiný materiál}
V~dokumentu lze používat mnoho a~mnoho dalších příkazů, které jsou
předdefinovány v~systému \TeX{} a~jeho nadstavbách, případně v~dalších
stylech, které si uživatel může připojit v~preambuli. Všechny příkazy lze
navíc změnit nebo rozšířit o~vlastní funkce -- k~tomu slouží příkaz pro
tvorbu nových příkazů \verb.\def. a~jemu podobné.

\kapitola{Závěr}
Pro usnadnění práce při typografické úpravě díla s~knižním písmem byl vyvinut
sazební styl \texttt{xdipp.sty} pro typografický systém \TeX{} a~jeho nadstavbu
\XeLaTeX. Tento styl je volně šiřitelný podobně jako celý zmíněný systém a~jeho
cílem je zvýšit estetickou i~technickou hodnotu závěrečných prací.

Jistým důkazem použitelnosti a~ukázkou výstupního tvaru generovaného vyrobeným
stylem je tento samotný text, v~němž byla použita následující preambule:

\begin{verbatim}
\documentclass[twoside,12pt]{article}%
\usepackage{xdipp}
\pismo{Constantia}
\popiskyzkr
\cislovat{2}
\begin{document}
...
\end{verbatim}

Samotný styl je přístupný na online sázecím systému \TeX onWeb, jehož adresa
je {\ttfamily tex.mendelu.cz}. Systém umožňuje celé zpracování uživatelských
dokumentů vzdáleně instalovaným systémem \XeLaTeX, k~němuž je standardně
k~dispozici příslušný sazební styl \verb!xdipp.sty!. Uživatel pouze připraví
online editorem (nebo jakýmkoliv vlastním) zdrojový text, který se po odeslání
na serveru přeloží, převede do standardní podoby PDF a~vrátí zpět. Komunikaci
v~tomto případě zajišťuje standardní webový prohlížeč. Na lokálním stroji
nemusí být kromě webového prohlížeče a~čtečky PDF instalován žádný další
software.

Styl je možné libovolně modifikovat. Můžete si vytvářet vlastní příkazy,
rozšiřovat možnosti existujících a~přizpůsobovat vizuální podobu dokumentu
svým potřebám. Jediným omezením je, že modifikovaný styl nesmíte šířit pod
stejným názvem.

V~případě jakýchkoliv dotazů, námětů nebo nalezených chyb pište na adresu
\verb"rybicka@mendelu.cz".

\begin{literatura}
     % V Lora nejsou kapitálky, takže je nutné je vložit ručně
     \citace{beran}{Beran, 1994}{\autor{Beran, V.} \nazev{Typografický manuál.}
          Náchod: Nakladatelství Manuál, 1994. \\ ISBN 80-901824-0-2}

     \citace{csn010166}{ČSN 01\,0166, 1992}{\nazev{ČSN 01\,0166 Nakladatelská
               (vydavatelská) úprava knih a~některých dalších druhů neperiodických
               publikací.} Praha: Federální úřad pro normalizaci a~měření, 1992}

     \citace{csn016910}{ČSN 01\,6910, 2014}{\nazev{ČSN 01\,6910 Úprava dokumentů
               zpracovaných textovými procesory.} Praha: ÚNMZ, 2014}

     \citace{csniso690}{ČSN ISO 690, 2011}{\nazev{ČSN ISO 690 -- Bibliografické
               citace. Obsah, forma a~struktura}. Praha: ÚNMZ, 2011}

     \citace{csn7144}{ČSN ISO 7144, 1996}{\nazev{ČSN ISO 7144 Dokumentace --
               Formální úprava disertací a~podobných dokumentů.} Praha: Český normalizační
          institut, 1996}

     \citace{csn80000}{ČSN ISO 80000-1}{\nazev{ČSN ISO 80000-1 Veličiny a~jednotky --
               Část 1: Obecně}. Praha: ÚNMZ, 2011}

     \citace{csn80000m}{ČSN ISO 80000-2}{\nazev{ČSN ISO 80000-2 Veličiny a~jednotky --
               Část 2: Matematická znaménka a~značky pro použití ve fyzikálních vědách
               a~v~technice}. Praha: ÚNMZ, 2011}

     \citace{Nohel}{Nohel, 1972}{\autor{Nohel, F.} \nazev{Sazba matematická
               a~chemická}. Praha: SNTL, 1972}

     \citace{pop}{Pop, Fléger a~Pop, 1989}{\autor{Pop, P., Flégr, J., Pop, V.}
          \nazev{Sazba I~-- Ruční sazba.} Praha: SPN, 1989}

     \citace{pravidla}{ÚJČ, 2019}{\autor{ÚJČ}. \nazev{Internetová jazyková příručka}
          [online] [vid. 10.\,3.\,2019]. Dostupné z: \texttt{prirucka.ujc.cas.cz}}

     \citace{latzac}{Rybička, 2003}{\autor{Rybička, J.} \nazev{\LaTeX{} pro
               začátečníky.} 3.\,vyd. Brno: Konvoj, 2003}

     \citace{TalDipl}{Talandová, 2006}{\autor{Talandová, P.} \nazev{Přístupy ve
               zpracování tabulek v~systémech DTP}. Diplomová práce. Brno: PEF MZLU v~Brně,
          2006}

     \citace{texonweb}{TeXonWeb, 2019}{\nazev{\TeX onWeb} [online] [vid. 12.\,3.\,2019].
          Dostupné z: https://tex.mendelu.cz/new}

\end{literatura}

\prilohy
\priloha{Určení rozsahu této práce}
\label{rozsah}
Vyjdeme ze dvou údajů, zjištěných měřením dokumentu: počet znaků s~mezerami
a~tisková plocha obrázků v~cm$^2$.

Počet znaků s~mezerami $z$: 36\,145

Plocha dvou obrázků $p$: $47 + 36 = 83$ cm$^2$

Počet AA $v$: $$v=\frac z{36\,000}+\frac p{2\,300} = 1,04\;\hbox{AA}$$

Vyjádříme-li rozsah v~normalizovaných stranách $n$, dostáváme:
$$n=v\cdot 20 \doteq 21\;\hbox{NS}$$

Tato práce má tedy rozsah přibližně 21 normalizovaných stran.

Vidíme, že počet fyzických stran se blíží počtu normalizovaných stran, i~když
je zvolen zcela odlišný formát sazby -- stránky vůbec neobsahují počet znaků
odpovídajících normalizované straně. Z~toho jasně vyplývá, že usuzovat na
rozsah díla podle počtu fyzických stran je velmi nepřesné a~mnohdy zcela zavádějící.
\end{document}
