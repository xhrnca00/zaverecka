\documentclass[main.tex]{subfiles}

\begin{document}
\kapitola{OOP a Rust}

Objektově orientované programování se hodně liší od C++. Podle některých definic OOP není
Rust ani objektově orientovaným jazykem \cite[kapitola\,17]{thebook}.

Již jsme viděli, že je možné vytvořit strukturu a dát jí nějaké metody. Objekty tedy
máme, ale jaké jsou vztahy mezi nimi? \emph{Rust nemá objektovou dědičnost}~-- jeden
struct nemůže dědit z~jiného. Žádné klíčové slovo class ale neexistuje.

\sekce{Trait systém}

Rust odděluje funkcionalitu od dat. Data jsou definována přímo ve~struktuře. Funkcionalita
spojená pouze s~daným typem je také definována přímo na něm. Když ale chceme nadefinovat
sdílenou funkcionalitu mezi typy, použijeme klíčové slovo \irust{trait}.

Trait je něco jako interface v~dalších objektově orientovaných jazycích. Definuje
asociované funkce (jejich hlavičky, případně i defaultní implementaci), nikoliv však
proměnné. \cite[sekce\,10.2]{thebook}

Výhodou je, že můžeme implementovat námi vytvořený trait pro cizí strukturu a obráceně.
Když tedy potřebujeme sdílenou funkcionalitu u~typu, který není náš, \emph{můžeme mu ji
    přidat}. Například existuje trait \irust{FromStr}, který definuje, jak máme z~řetězce
vytvořit daný typ. Když tedy chceme umět parsovat nějaký náš typ, stačí nám implementovat
tento trait (nikoliv přetížením \icpp{std::istream& operator>>}).
\cite[str/trait.FromStr]{ruststd}

Díky tomuto také najdeme každou funkci, která má co dělat s~daným typem, přímo jako její
metodu (toto platí i~u~primitivních typů jako \irust{i32}). Náš našeptávač je tedy o~dost
víc nápomocný než u~C++.

Příklad vytvoření traitu a jeho implementace viz obr.\,\ref{trait creation showcase}

\obrazek
\begin{rustcode}
    use std::f64::consts::PI;

    trait Shape {
        fn area(&self) -> f64;
    }

    struct Rectangle {
        width: f64,
        height: f64,
    }

    struct Circle {
        radius: f64,
    }

    impl Shape for Rectangle {
        fn area(&self) -> f64 {
            self.width * self.height
        }
    }

    impl Shape for Circle {
        fn area(&self) -> f64 {
            self.radius * self.radius * PI
        }
    }
\end{rustcode}
\endobrl{Ukázka používání traitu}{trait creation showcase}

\sekce{Trait bound a generiky}

Pokud chceme v~generické funkci použít nějakou funkcionalitu, musíme vědět, že typ ji má
(program se jinak ani nezkompiluje). Toho docílíme tak, že na generický typ dáme tzv.
trait bound. To znamená, že aby mohl tento typ být použit, musí implementovat daný trait
(může jich být i~víc). \cite[sekce\,10.2]{thebook}

Tohoto lze docílit buď klíčovým slovem \irust{where}, nebo dvojtečkovým syntaxem přímo
v~parametru. Pro každý pevně daný typ se vygeneruje \emph{jiná} funkce, kde je za generiku
dosazena pevná hodnota \cite[sekce\,10.1]{thebook}.
Ukázka viz obr.\,\ref{generic function showcase}

\obrazek
\begin{rustcode}
    fn print_area_generic1<T: Shape>(shape: &T) {
        println!("This shape has an area of {:.2}", shape.area());
    }

    fn print_area_generic2<T>(shape: &T)
    where
        T: Shape,
    {
        println!("This shape has an area of {:.2}", shape.area());
    }
\end{rustcode}
\endobrl{Ukázka generických funkcí s~trait boundem}{generic function showcase}

\sekce{Trait objekty a dynamic dispatch}

Generiky ale mají jeden problém: můžeme je použít, když máme hodnoty pouze daného typu.
Co když ale máme kolekci ukazatelů na objekty, které všechny implementují daný
trait\footnote{ukazatelů proto, že každý typ může mít jinou velikost}? Použijeme tzv.
trait objekty.

Pokud používáme trait objekty, potřebujeme nějak zjistit, jakou implementaci máme zavolat.
Tomuto se říká dynamic dispatch~-- dynamicky za běhu programu zjišťujeme, kde se nachází
funkce, kterou máme zavolat. Proto také reference na trait objekty označujeme slovem
\irust{dyn}. \cite[sekce\,17.2]{thebook} Příklad funkce ekvivalentní s~generickými na
obr.\,\ref{trait object showcase}

\obrazek
\begin{rustcode}
    fn print_area_tobj(shape: &dyn Shape) {
        println!("This shape has an area of {:.2}", shape.area());
    }
\end{rustcode}
\endobrl{Ukázka trait objektů}{trait object showcase}

\sekce{Derivační makra}

Některé traity nemusíme definovat manuálně, může to za nás udělat derivační makro. Stačí
nám pouze nad definici struktury napsat \irust{#[derive(TraitName)]} (do závorek můžeme
napsat více názvů, které oddělíme čárkou). \cite[sekce\,5.2]{thebook}

Tuto funkcionalitu nemá každý trait. Mezi ty, které ji mají patří například
\irust{Debug} (umožňující tisknutí pro debugování), \irust{Clone} (hluboké kopírování),
\irust{PartialEq} (částečné porovnávání) a další. \cite[Appendix C]{thebook}

\sekce{Nejpoužívanější traity}

\begin{itemize}
    \item \irust{Display}~-- tisknutí ve formátovaných řetězcích
    \item \irust{Debug}~-- tisknutí struktury objektu ve formátovaných řetězcích
    \item \irust{Iterator}~-- všechny iterátory jej implementují, díky němu lze použít
          tzv. builder pattern na iterátorech
    \item \irust{Error}~-- požadavky pro errory
    \item \irust{PartialEq}~-- částečné porovnání (\irust{==}) (nezaručena ekvivalence)
    \item \irust{PartialOrd}~-- částečné srovnávání (\irust{< > <= >=})
    \item \irust{Copy}~-- kopírovací sémantika
    \item \irust{Clone}~-- hluboká kopie (včetně obsahu ukazatelů), \uv{duplikace}
    \item \irust{Sized}~-- velikost dané proměnné je možné určit při kompilaci
    \item \irust{Sync}~-- typy, jejichž reference je bezpečné sdílet mezi vlákny
    \item \irust{Send}~-- bezpečnost posílání typů mezi vlákny
    \item \irust{FromStr}~-- parsování hodnoty z~řetězce
    \item \irust{IntoIterator}~-- převedení na iterátor
    \item \irust{From<T>}~-- převedení \emph{z}~hodnoty~T
    \item \irust{Into<T>}~-- prevedení \emph{na}~hodnotu~T
\end{itemize}
Každý operátor má také vlastní trait. Díky tomu není potřeba žádný zvláštní syntax pro
přetěžování operátorů~-- stačí pouze implementovat daný trait (např. \irust{Add} pro
sčítání). \cite[sekce\,19.2]{thebook}

\sekce{Generické implementace~-- automatické rozšiřování}

Pomocí generik můžeme také implementovat trait pro každý typ splňující generiku. Toto je
velmi užitečné, například knihovna \texttt{itertools} dělá právě to~-- rozšiřuje
\irust{Iterator} trait o~další metody \cite[itertools, src/itertools/lib.rs]{docsrs}. Také
toho využívají traity \irust{From} a \irust{Into}~-- pokud implementujeme \irust{From},
druhá implementace se vytvoří automaticky (pokud umíme vytvořit \irust{T} z~\irust{E},
umíme konvertovat \irust{E} na~\irust{T}) \cite[convert/trait.From]{ruststd}.

\obrazek
\begin{rustcode*}{highlightlines={17-17}}
    trait Average: Iterator<Item = usize> {
        fn average(mut self) -> f64
        where
            Self: Sized,
        {
            let mut sum = 0;
            let mut count: usize = 0;
            for value in self {
                sum += value;
                count += 1;
            }
            // will be NaN if the iterator is empty
            sum as f64 / count as f64
        }
    }

    impl<T: Iterator<Item = usize>> Average for T {}

    fn main() {
        let v = vec![1, 2, 3];
        println!("{}", v.into_iter().average());
    }
\end{rustcode*}
\endobrl{
    Ukázka rozšíření funkcionality pomocí generických implementací
}{generic impl showcase}

V~ukázce definujeme tzv. supertrait \irust{Average}. Supertrait je to proto, že aby jej
nějaký typ implementoval, musí už implementovat trait \irust{Iterator}, jehož asociovaný
typ \irust{Item} je \irust{usize} (toto je řečeno za dvojtečkou po názvu supertraitu)~--
iteruje tedy prvky typu \irust{usize}.

\irust{Average} definuje jedinou funkci \irust{average}, která konzumuje \irust{self}.
Protože nebereme \irust{self} přes referenci, musíme vědět jeho velikost při kompilaci
(proto \irust{Sized}). Tato funkce má defaultní implementaci, která spočítá průměr prvků
v~iterátoru. Ve~funkci iterujeme \irust{self} (je to uričtě iterátor), proto taky
potřebujeme \irust{self} měnit (a je tedy označené \irust{mut}). Typ obou použitých
proměnných je \irust{usize}, musíme je tedy převést na \irust{f64} pomocí klíčového slova
\irust{as}.

Na označeném řádku implementujeme \irust{Average} pro všechny iterátory, jejichž prvek je
\irust{usize}. V~\irust{main} funkci tiskneme průměr čísel ve~vektoru \irust{v}. Typ
proměnné \irust{v} je \irust{Vec<usize>} (kompilátor vidí, že jiná možnost není).

\end{document}
