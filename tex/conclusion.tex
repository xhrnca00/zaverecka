\documentclass[main.tex]{subfiles}

\begin{document}
\kapitola{Závěr}

Cílem práce bylo představit programovací jazyk Rust. Tento cíl byl splněn. Práce by šla
rozšířit představením většího projektu v~Rustu, kde by se prakticky využily získané
znalosti a lépe projevila schopnost jazyka elegantně řešit časté problémy jako například
zpracování chyb.

Rust je dobrou náhradou za C++, jdou v~něm vytvářet bezpečnější programy a zároveň je
uživatelsky přívětivější díky lepším nástrojům. Celosvětová komunita má rovněž tento
trend, spousta velkých společností volí pro nové projekty exkluzivně Rust místo C++
(například Microsoft, Amazon,\dots). Doporučuji si tedy minimálně s~tímto jazykem
pohrát~-- třeba mu přijdete na chuť.

Také si myslím, že se jedná o~skvělý učící nástroj pro paralelní programování~--
kompilátor totiž hlídá bezpečnost a \uv{správnost} programu (nestane se tedy, že někde
zapomeneme na \irust{Mutex} objekt). Osobně jsem dokázal za cca 30\,minut vytvořit
komplikovanější program, aniž bych měl dřívější zkušenost s~paralelním programováním.

\end{document}
